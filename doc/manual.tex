\documentclass[.52pt,a4paper,titlepage]{article}

\usepackage{a4wide}
\usepackage[T1]{fontenc}
\usepackage{hyperref}

\usepackage{tikz}
\usepackage{circuitikz}
\usepackage{ctikzmanutils}
\usepackage[]{csquotes}
\usetikzlibrary{calc, positioning, automata, arrows.meta, riscvproc}

\usepackage{caption}
\usepackage{fancyhdr} 
\usepackage{lmodern}
\usepackage{showexpl}
\usepackage{tabularx}
\usepackage{xcolor}

\hypersetup{
	pdfauthor={Niklas Ladurner},
	pdftitle={A RISC-V Processor Components \Circuitikz{} Library},
	pdfsubject={},
	pdfkeywords={riscv, processor, circuitikz},
	pdfcreator={pdflatex},
	colorlinks=false,
}


\renewcommand{\rmdefault}{lmss} % set font to sans serif

\parindent=0pt
\parskip=4pt plus 6pt minus 2pt

\lstset{
	gobble=6,
	basicstyle=\ttfamily\small
}

\definecolor{modblue}{HTML}{288DCC}

% modified https://github.com/circuitikz/circuitikz/blob/c94be92df781cfbcba21c6d9c5e95a98976917a5/doc/ctikzmanutils.sty#L108
\newcommand{\widetwopartbox}[2]{%
	\leavevmode\null\par\noindent\fbox{\parbox[c]{0.4\linewidth}{#1} \parbox[c]{0.6\linewidth}{\RaggedRight\hbadness=9500 #2}\par\noindent}%
}
\NewDocumentCommand{\widecircuitdesc}{s O{1} m m m d() d[]}
{
	\widetwopartbox{%
		\centering
		\begin{circuitikz}[]
			\IfBooleanTF{#1}{%
				\draw (0,0) node[#3,scale=#2, fill=fillcol](N){#5};
			}{
				% if it's non-fillable, red should not go through
				\draw (0,0) node[#3,scale=#2,
				% fill=red
				](N){#5};
			}
			\IfValueT{#6}{%
				\foreach \n/\a/\d in {#6} \path(N.\n) \showcoord(\n)<\a:\d>;
			}
			\IfValueT{#7}{%
				\foreach \n/\a/\d in {#7} \path(N-\n) \showcoordb(N-\n)<\a:\d>;
			}
		\end{circuitikz}%
	}{\sloppy
		{#4, type: node\IfBooleanT{#1}{, fillable}%
		} (\texttt{node[\detokenize{#3}]\IfValueT{#7}{(N)}\{\detokenize{#5}\}}). \index{#3}%
		\checkclass{N}%
	}%
}

\title{A RISC-V Processor Components \Circuitikz{} Library}
\author{Niklas Ladurner}
\date{\today}

\begin{document}

\begin{center}
	\LARGE \textbf{\thetitle}

	\normalsize \theauthor

	\normalsize \thedate
\end{center}

\tableofcontents

\section{Introduction}
\subsection{Motivation}
This \Circuitikz{} library offers some components to efficiently draw RISC-V processors in \LaTeX. The library was designed with the goal of resembling the RISC-V processor schematics as presented in \enquote*{Digital Design and Computer Architecture: RISC-V Edition} by Sarah L. Harris and David Harris.
\subsection{Usage}
To use the predefined components, you must include the library \texttt{riscvproc}. Your preamble should look like this:
\begin{verbatim}
	...
	\usepackage{tikz}
	\usepackage{circuitikz}
	\usetikzlibrary{riscvproc}
	...
\end{verbatim}

Components are then availiable in \texttt{circuitikz} environments:
\begin{center}
	\begin{LTXexample}[varwidth, rframe=]
		\begin{circuitikz}
			\node[instrmem, align=center] (comp) {Instruction\\Memory};
			\draw[->, red] (comp.a) -- ++(-.5, 0) node[left] {a};
			\draw[->, blue] (comp.rd) -- ++(.5, 0) node[right] {rd};
		\end{circuitikz}
	\end{LTXexample}
\end{center}

\newpage
\section{Component List}
\subsection{Memory Components}
\widecircuitdesc*{instrmem}{Instruction memory}{}(a/180/0.2, rd/0/0.2, ba/135/0.2, brd/45/0.2)

\widecircuitdesc*{datamem}{Data Memory}{}(a/180/0.2, wd/180/0.2, rd/0/0.2, clk/90/0.2,we/90/0.2, ba/135/0.2, bwd/225/0.2, brd/45/0.2, bclk/135/0.2,bwe/45/0.2)

\widecircuitdesc*{regfile}{Register File}{}(
a1/180/0.2,ba1/135/0.2,
a2/180/0.2,ba2/135/0.05,
a3/180/0.2,ba3/225/0.05,
wd3/180/0.2,bwd3/225/0.2,
rd1/0/0.2,brd1/45/0.2,
rd2/0/0.2,brd2/-45/0.2,
clk/90/0.2,bclk/135/0.2,
we3/90/0.2,bwe3/45/0.2)

\subsection{Arithmetic Components}
\widecircuitdesc*{alu}{Arithmetic Logic Unit}{}(
a/180/0.2,ba/135/0.2,
b/180/0.2,bb/225/0.2,
out/0/0.2,bout/-45/0.2,
zero/0/0.2,bzero/45/0.2,
ctrl/90/0.2,bctrl/45/0.2)


\widecircuitdesc*{adder}{Adder}{}(
a/180/0.2,ba/135/0.2,
b/180/0.2,bb/225/0.2,
out/0/0.2,bout/45/0.2)


\widecircuitdesc*{subtr}{Subtractor}{}(
a/180/0.2,ba/135/0.2,
b/180/0.2,bb/225/0.2,
out/0/0.2,bout/45/0.2)

\subsection{Multiplexers}
\widecircuitdesc*{mux}{Multiplexer}{}(
in0/180/0.2,bin0/135/0.2,
in1/180/0.2,bin1/225/0.2,
sel/90/0.2,bsel/45/0.2,
out/0/0.2,bout/45/0.2)

\widecircuitdesc*{3mux}{Multiplexer with 3 inputs}{}(
in0/180/0.2,bin0/135/0.2,
in1/180/0.2,bin1/225/0.025,
in2/180/0.2,bin2/225/0.2,
sel/90/0.2,bsel/45/0.2,
out/0/0.2,bout/45/0.2)

\widecircuitdesc*{4mux}{Multiplexer with 4 inputs}{}(
in0/180/0.2,bin0/135/0.2,
in1/180/0.2,bin1/135/0.025,
in2/180/0.2,bin2/225/0.025,
in3/180/0.2,bin3/225/0.2,
sel/90/0.2,bsel/45/0.2,
out/0/0.2,bout/45/0.2)

\subsection{Control Units}
\widecircuitdesc*{ctrlunitsc}{Single-Cycle Control Unit}{}(
op/180/0.2,
funct3/180/0.2,
funct7/180/0.2,
zero/180/0.2,
pcsrc/0/0.2,
resultsrc/0/0.2,
memwrite/0/0.2,
alucontrol/0/0.2,
alusrc/0/0.2,
immsrc/0/0.2,
regwrite/0/0.2)

\widecircuitdesc*{ctrlunitsc}{Single-Cycle Control Unit (b-Anchors)}{}(
bop/180/0.2,
bfunct3/180/0.2,
bfunct7/180/0.2,
bzero/180/0.2,
bpcsrc/0/0.2,
bresultsrc/0/0.2,
bmemwrite/0/0.2,
balucontrol/0/0.2,
balusrc/0/0.2,
bimmsrc/0/0.2,
bregwrite/0/0.2)

\widecircuitdesc*{ctrlunitmc}{Multi-Cycle Control Unit}{}(
clk/90/0.2,
op/180/0.2,
funct3/180/0.2,
funct7/180/0.2,
zero/180/0.2,
pcwrite/180/0.2,
adrsrc/180/0.2,
memwrite/180/0.2,
irwrite/180/0.2,
resultsrc/0/0.2,
alucontrol/0/0.2,
alusrca/0/0.2,
alusrcb/0/0.2,
immsrc/0/0.2,
regwrite/0/0.2)

\widecircuitdesc*{ctrlunitmc}{Multi-Cycle Control Unit (b-Anchors)}{}(
bclk/90/0.2,
bop/180/0.2,
bfunct3/180/0.2,
bfunct7/180/0.2,
bzero/180/0.2,
bpcwrite/180/0.2,
badrsrc/180/0.2,
bmemwrite/180/0.2,
birwrite/180/0.2,
bresultsrc/0/0.2,
balucontrol/0/0.2,
balusrca/0/0.2,
balusrcb/0/0.2,
bimmsrc/0/0.2,
bregwrite/0/0.2)

\widecircuitdesc*{ctrlunitpl}{Pipelined Control Unit}{}(
op/180/0.2,
funct3/180/0.2,
funct7/180/0.2,
regwrite/0/0.2,
resultsrc/0/0.2,
memwrite/0/0.2,
jump/0/0.2,
branch/0/0.2,
alucontrol/0/0.2,
alusrc/0/0.2,
immsrc/0/0.2)

\widecircuitdesc*{ctrlunitpl}{Pipelined Control Unit (b-Anchors)}{}(
bop/180/0.2,
bfunct3/180/0.2,
bfunct7/180/0.2,
bregwrite/0/0.2,
bresultsrc/0/0.2,
bmemwrite/0/0.2,
bjump/0/0.2,
bbranch/0/0.2,
balucontrol/0/0.2,
balusrc/0/0.2,
bimmsrc/0/0.2)

\subsection{Miscellaneous Components}
\widecircuitdesc*{extend}{Extend Unit}{}(
in/180/0.2,bin/135/0.2,
out/0/0.2,bout/45/0.2,
ctrl/90/0.2,bctrl/135/0.2)

\widecircuitdesc*{reg}{Register}{}(
in/180/0.2,bin/135/0.2,
out/0/0.2,bout/45/0.2,
clk/90/0.2,bclk/135/0.2)


\section{Keys}
\subsection{\Circuitikz{} keys}
The desired \Circuitikz{} key can be set via \verb|\ctikzset{processor/<key>=value}|. E.g. if one whishes to set the line width of all components to 4, the line  \verb|\ctikzset{processor/thickness=4}| would have to be included in the specific circuitikz picture. A list of all \Circuitikz{} keys can be found in Table \ref{tab:keys}. A list of component families can be found in Table \ref{tab:families}.

\begin{figure}[htbp]
	\begin{tabularx}{\textwidth}{|lXr|}
		\hline
		Key                       & Description                                                           & Default value    \\
		\hline
		\texttt{scale}            & Sets scale for all processor components.                              & \texttt{1}       \\
		\texttt{thickness}        & Sets line width for all processor components.                         & \texttt{2}       \\
		\texttt{font}             & Sets font family for all labels of processor components.              & \verb|\rmfamily| \\
		\texttt{memory/height}    & Sets height for all memory components.                                & \texttt{2}       \\
		\texttt{memory/width}     & Sets width for all memory components except \texttt{regfile}.         & \texttt{1.25}    \\
		\texttt{control/heightsc} & Sets height for \texttt{ctrlunitsc}.                                  & \texttt{2.5}     \\
		\texttt{control/heightmc} & Sets height for \texttt{ctrlunitmc}.                                  & \texttt{3.5}     \\
		\texttt{control/width}    & Sets width for control components.                                    & \texttt{0.9}     \\
		\texttt{control/radius}   & Sets border radius for control components.                            & \texttt{5}       \\
		\texttt{arith/height}     & Sets height for arithmetic components.                                & \texttt{0.9}     \\
		\texttt{arith/width}      & Sets height for arithmetic components.                                & \texttt{0.7}     \\
		\texttt{arith/slope}      & Sets slope for arithmetic components in degrees.                      & \texttt{15}      \\
		\texttt{extend/height}    & Sets height for big side of extend components.                        & \texttt{0.6}     \\
		\texttt{extend/width}     & Sets height for extend components.                                    & \texttt{2}       \\
		\texttt{extend/slope}     & Sets slope for extend components in degrees.                          & \texttt{7}       \\
		\texttt{mux/slope}        & Sets slope for multiplexers in degrees.                               & \texttt{15}      \\
		\texttt{misc/smallheight} & Sets height for small components.                                     & \texttt{0.65}    \\
		\texttt{misc/smallwidth}  & Sets width for small components. Also affects the CLK input triangle. & \texttt{0.3}     \\
		\texttt{misc/leadlen}     & Sets length for input and output leads.                               & \texttt{0.25}    \\
		\hline
	\end{tabularx}
	\captionof{table}{List of \Circuitikz{} keys}
	\label{tab:keys}
\end{figure}

\begin{figure}[htbp]
	\begin{tabularx}{\textwidth}{|lX|}
		\hline
		Component family      & Component list                                        \\
		\hline
		memory components     & \texttt{instrmem}, \texttt{datamem}, \texttt{regfile} \\
		control components    & \texttt{ctrlunitsc}, \texttt{ctrlunitmc}              \\
		arithmetic components & \texttt{alu}, \texttt{add}, \texttt{subtr}            \\
		extend components     & \texttt{extend}                                       \\
		small components      & \texttt{mux}, \texttt{reg}                            \\
		\hline
	\end{tabularx}
	\captionof{table}{List of component families}
	\label{tab:families}
\end{figure}
\subsection{Special node keys}
Some keys are also defined as  Tikz keys and can therefore be directly passed to nodes likes shown in Figure \ref{ex:keys}. A list of all these keys can be found in Table \ref{tab:tikz_keys}.

\begin{figure}[t!]
	\begin{LTXexample}[varwidth, rframe=]
		\begin{circuitikz}
			\node[reg, align=center, stacks=2, no output leads, enable input] (comp) {};
		\end{circuitikz}
	\end{LTXexample}
	\caption{Passing options to a node}
	\label{ex:keys}
\end{figure}
\begin{figure}[t!]
	\begin{tabularx}{\textwidth}{|lXr|}
		\hline
		Key                   & Description                                                                                                     & applicable to  \\
		\hline
		\texttt{input leads}  & Specifies wether to draw input leads.                                                                           & all components \\
		\texttt{output leads} & Specifies wether to draw output leads.                                                                          & all components \\
		\texttt{leads}        & Specifies wether to draw leads at all.                                                                          & all components \\
		\texttt{stacks}       & Sets height of a register in multiples of the default height, allows for stretched registers.                   & \texttt{reg}   \\
		\texttt{enable input} & Specifies wether to draw an enable input or not. This also gives two new anchors, \texttt{en} and \texttt{ben}. & \texttt{reg}   \\
		\texttt{clock} & Specifies wether to draw a clk input on a component that supports it. & all timed components  \\

		\hline
	\end{tabularx}
	\captionof{table}{List special node keys}
	\label{tab:tikz_keys}
\end{figure}

\vspace{\baselineskip}
More keys might be added in future.

\newpage
\section{Examples}
\subsection{Single-Cycle RISC-V Processor}
\begin{center}
	\resizebox*{\textwidth}{!}{
		\begin{circuitikz}
	\tikzset{
		label/.style={draw=none, inner sep=1pt, font=\small},
		comp/.style={align=center, no leads},
		custyle/.style={color=modblue},
	}
	
	% components
	\node[regfile, comp] (rf) {Register\\File};
	\node[instrmem, comp, left=2cm of rf] (im) {Instruction\\Memory};
	\node[alu,comp, right=2.25cm of rf.rd1, anchor=a] (alu) {};
	\node[datamem,comp, right=6.25cm of rf] (dm) {Data\\Memory};
	
	\node[ctrlunitsc, comp, custyle, above=1cm of rf, xshift=-0.75cm] (cu) {Control\\Unit};
	
	\node[extend, comp, below=1cm of rf] (ext) {Extend};
	\node[adder, comp, anchor=b] (addpctarget) at (ext.out -| alu.a) {};
	\node[adder, comp, anchor=a] (addpcplus4) at (ext.in -| im.a) {};
	
	% TODO: align mux.out and alu.b
	\node[mux, comp, right=0.9cm of rf.rd2, anchor=in0] (muxalusrc) {};
	\node[reg, comp, left=0.3cm of im.a] (regpcnext) {};
	\node[mux, comp, left=1.25cm of regpcnext] (muxpcnext) {};
	\node[3mux, comp, right=1cm of dm.rd, anchor=in1] (muxresult) {};
	
	% special nodes
	% clks
	\node[align=center, label, above=0.25cm of regpcnext.clk] (clk1) {CLK};
	\draw[] (clk1) -- (regpcnext.clk);
	\node[align=center, label, above=0.25cm of rf.bclk] (clkrf) {CLK};
	\draw[] (clkrf) -- (rf.bclk);
	\node[align=center, label, above=0.25cm of dm.bclk] (clkdm) {CLK};
	\draw[] (clkdm) -- (dm.bclk);
	
	% constants
	\node[align=right, label, left=0.25cm of addpcplus4.bb] (const4) {4};
	\draw[] (const4) -- (addpcplus4.bb);
	
	% connections
	% register file
	\draw[] (rf.brd1) -- (alu.ba);
	\draw[] (rf.brd2) -- (muxalusrc.bin0);
	\draw[] (rf.brd2) -- ++(1, 0) |- (dm.bwd);
	
	% alu
	\draw[] (muxalusrc.bout) |- (alu.bb);
	\draw[] (alu.bout) -| (dm.ba);
	\node[] (anchorzero) at ($(alu.bout) + (1,1.75)$) {};
	\node[right=0.5cm of anchorzero] (anchoralures){};
	\draw[] (alu.bzero) -| (anchorzero.center) -- ++(-8,0)  |- (cu.bzero);
	\draw[] (alu.bout) -| (anchoralures.center)  -- ++(3.5,0) |- (muxresult.bin0);
	
	% data memory
	\draw[] (dm.brd) -- (muxresult.bin1);
	\draw[] (muxresult.bout) -- ++(0.25,0) -- ++(0,-5) -|  node[align=center,above,pos=0.02] {Result} ($(rf.bwd3) + (-1.25,0)$) -- (rf.bwd3);
	
	% extend unit
	\draw[] (ext.bout) -- (addpctarget.bb);
	\draw[] (ext.bout) ++(1.5, 0)|- (muxalusrc.bin1);
	
	% instruction memory
	\draw[] (im.brd) -- ++(1.15, 0) |- (cu.bop);
	\draw[] (im.brd) -- ++(1.15, 0) |- (cu.bfunct3);
	\draw[] (im.brd) -- ++(1.15, 0) |- (cu.bfunct7);
	\draw[] (im.brd) -- ++(1.15, 0) |- (rf.ba1);
	\draw[] (im.brd) -- ++(1.15, 0) |- (rf.ba2);
	\draw[] (im.brd) -- ++(1.15, 0) |- (rf.ba3);
	\draw[] (im.brd) -- ++(1.15, 0) |- (ext.bin);
	\draw[] (regpcnext.bout) -- (im.ba);
	
	% control unit
	\draw[custyle] (cu.bregwrite) -| (rf.bwe3);
	\draw[custyle] (cu.bimmsrc) -- ++(2, 0) -- ++(0, -5.5) -| (ext.bctrl);
	\draw[custyle] (cu.balusrc) -| (muxalusrc.bsel);
	\draw[custyle] (cu.balucontrol) -| (alu.bctrl);
	\draw[custyle] (cu.bmemwrite) -| (dm.bwe);
	\draw[custyle] (cu.bresultsrc) -| (muxresult.bsel);
	\draw[custyle] (cu.bpcsrc) -- ++(1.25, 0) -- ++(0, 0.75)-| (muxpcnext.bsel);
	
	% additional connections
	\draw[] (addpcplus4.bout) -| ($(muxresult.bin2) + (-0.5, 0)$) -- (muxresult.bin2);
	\draw[] (addpcplus4.bout) -- ++(0.25,0) -- ++(0, -1.25)  -- ++ (-4.5,0) |- (muxpcnext.bin0);
	\draw[] (regpcnext.bout) -- ++(0.3, 0) |- (addpcplus4.ba);
	\draw[] (regpcnext.bout) -- ++(0.3, 0) |- (addpctarget.ba);
	\draw[] (addpctarget.bout) -- ++(0.25, 0) -- ++(0, -2.75) -- ++ (-14.5,0) |- (muxpcnext.bin1);
	\draw[] (muxpcnext.bout) -- (regpcnext.bin) node[align=center, label, above, pos=0.5] {PCNext};
	
	% labels
	\node[align=right, label, left=0.25em of alu.ba, yshift=0.5em] {SrcA};
	\node[align=right, label, left=0.25em of alu.bb, yshift=0.5em] {SrcB};
	\node[align=left, label, right=0.25em of alu.bout, yshift=0.5em] {ALUResult};
	\node[align=left, label, right=0.25em of alu.bzero, yshift=0.5em] {Zero};
	\node[align=left, label, right=0.25em of dm.brd, yshift=0.5em] {ReadData};
	\node[align=left, label, right=0.25em of alu.bout, yshift=-1.15cm+0.5em] {WriteData};
	\node[align=left, label, right=0.25em of ext.bout, yshift=0.5em] {ImmExt};
	\node[align=right, label, left=0.25em of cu.bop, yshift=0.5em] {\scriptsize{6:0}};
	\node[align=right, label, left=0.25em of cu.bfunct3, yshift=0.5em] {\scriptsize{14:12}};
	\node[align=right, label, left=0.25em of cu.bfunct7, yshift=0.5em] {\scriptsize{30}};
	\node[align=right, label, left=0.25em of rf.ba1, yshift=0.5em] {\scriptsize{19:15}};
	\node[align=right, label, left=0.25em of rf.ba2, yshift=0.5em] {\scriptsize{24:20}};
	\node[align=right, label, left=0.25em of rf.ba3, yshift=0.5em] {\scriptsize{11:7}};
	\node[align=right, label, left=0.25em of ext.bin, yshift=0.5em] {\scriptsize{31:7}};
	\node[align=left, label, custyle, right=0.25em of cu.bregwrite, yshift=0.5em] {RegWrite};
	\node[align=left, label, custyle, right=0.25em of cu.bimmsrc, yshift=0.5em] {ImmSrc\textsubscript{1:0}};
	\node[align=left, label, custyle, right=0.25em of cu.balusrc, yshift=0.5em] {ALUSrc};
	\node[align=left, label, custyle, right=0.25em of cu.balucontrol, yshift=0.5em] {ALUControl\textsubscript{2:0}};
	\node[align=left, label, custyle, right=0.25em of cu.bmemwrite, yshift=0.5em] {MemWrite};
	\node[align=left, label, custyle, right=0.25em of cu.bresultsrc, yshift=0.5em] {ResultSrc\textsubscript{1:0}};
	\node[align=left, label, custyle, right=0.25em of cu.bpcsrc, yshift=0.5em] {PCSrc};
	\node[align=left, label, right=0.25em of addpcplus4.bout, yshift=0.5em] {PCPlus4};
	\node[align=left, label, right=0.25em of addpctarget.bout, yshift=0.5em] {PCTarget};
	\node[align=left, label, right=0.25em of im.brd, yshift=0.5em] {Instr};
	\node[align=left, label, right=0.25em of regpcnext.bout, yshift=0.5em] {PC};
\end{circuitikz}

	}
\end{center}


\subsection{Multi-Cycle RISC-V Processor}
\begin{center}
	\resizebox*{\textwidth}{!}{
		\begin{circuitikz}[]
	\tikzset{
		label/.style={draw=none, inner sep=1pt, font=\small},
		comp/.style={align=center, no leads},
		custyle/.style={color=modblue},
	}
	
	% components
	\node[regfile, comp] (rf) {Register\\File};
	\node[datamem, comp, left=4cm of rf] (dm) {Instr/Data\\Memory};
	\node[alu,comp, right=3.75cm of rf.rd1, anchor=a] (alu) {};
	
	\node[ctrlunitmc, comp, custyle, above=2.25cm of rf, xshift=-1.7cm] (cu) {Control\\Unit};
	
	\node[extend, comp, below=0.5cm of rf] (ext) {Extend};
	
	\node[reg, comp, right=2cm of rf.north, anchor=north, stacks=2] (regrfread) {}; % long reg next to rf
	
	\node[3mux, comp, right=1.75cm of rf.brd1, anchor=bin2] (muxalusrca) {};
	\node[3mux, comp, right=2.5cm of rf.brd2, anchor=bin0] (muxalusrcb) {};
	
	\node[mux, comp, left=0.75cm of dm.ba, anchor=bout] (muxaddrnext) {};
	\node[reg, comp, left=0.75cm of muxaddrnext.bin0, enable input, anchor=bout] (regpcnext) {}; % reg to store pcnext
	
	\node[reg, comp, right=1.8cm of alu.bout, anchor=bin] (regaluresult) {};
	\node[3mux, comp, right=1.5cm of regaluresult.bout, anchor=bin0] (muxresult) {};
	
	\node[reg, comp, right=0.75cm of dm.brd, enable input, anchor=bin, yshift=1.5cm, stacks=2.5] (regdmread) {}; % big reg next to dm
	\node[reg, comp, right=0.75cm of dm.brd, anchor=bin, yshift=-3.75cm,] (regdmread2) {}; % small reg below dm
	
	% special nodes
	% clks
	\node[align=center, label, above=0.25cm of regpcnext.clk] (clk1) {CLK};
	\draw[] (clk1) -- (regpcnext.clk);
	\node[align=center, label, above=0.25cm of regdmread.clk] (clk2) {CLK};
	\draw[] (clk2) -- (regdmread.clk);
	\node[align=center, label, above=0.25cm of regdmread2.clk] (clk3) {CLK};
	\draw[] (clk3) -- (regdmread2.clk);
	\node[align=center, label, above=0.25cm of regaluresult.clk] (clk4) {CLK};
	\draw[] (clk4) -- (regaluresult.clk);
	\node[align=center, label, above=0.25cm of regrfread.clk] (clk5) {CLK};
	\draw[] (clk5) -- (regrfread.clk);
	\node[align=center, label, above=0.25cm of rf.bclk] (clkrf) {CLK};
	\draw[] (clkrf) -- (rf.bclk);
	\node[align=center, label, above=0.25cm of dm.bclk] (clkdm) {CLK};
	\draw[] (clkdm) -- (dm.bclk);
	\node[align=center, label, custyle, above=0.25cm of cu.bclk] (clkcu) {CLK};
	\draw[custyle] (clkcu) -- (cu.bclk);
	
	% constants
	\node[align=right, label, left=0.25cm of muxalusrcb.bin2] (const4) {4};
	\draw[] (const4) -- (muxalusrcb.bin2);
	
	% connections
	% register file
	\draw[] (rf.brd1) -- (rf.brd1 -| regrfread.bin);
	\draw[] (rf.brd2) -- (rf.brd2 -| regrfread.bin);
	\draw[] (rf.brd2 -| regrfread.bout) -- ++(0.5, 0) -- ++(0, -1.75) -| ($(dm.bwd) + (-0.5, 0)$)-- (dm.bwd); % WriteData connection
	\draw[] (rf.brd1 -| regrfread.bout) -- (muxalusrca.bin2);
	\draw[] (rf.brd2 -| regrfread.bout) -- (muxalusrcb.bin0);
	
	% alu
	\draw[] (muxalusrca.bout) -| ($(alu.ba) + (-1, 0)$)-- (alu.ba);
	\draw[] (muxalusrcb.bout) -| ($(alu.bb) + (-1, 0)$)-- (alu.bb);
	\draw[] (alu.bout) -- (regaluresult.bin);
	\draw[] (alu.bzero) -- ++(1, 0) -- ++(0, 2.85) -| ($(cu.bzero) + (-0.5, 0)$)-- (cu.bzero);
	
	% instruction/data memory
	\draw[] (dm.brd) --++(0.3,0) |- (regdmread2.bin);
	\draw[] (dm.brd) --++(0.3,0) -- (dm.brd -| regdmread.bin);
	\draw[] (regdmread2.bout) -| ($(muxresult.bin1) + (-1, 0)$) -- (muxresult.bin1);
	\draw[] (muxaddrnext.bout) -- (dm.ba);
	
	\draw[] (dm.brd -| regdmread.bout) -- ++(1.1, 0) |- (cu.bop);
	\draw[] (dm.brd -| regdmread.bout) -- ++(1.1, 0) |- (cu.bfunct3);
	\draw[] (dm.brd -| regdmread.bout) -- ++(1.1, 0) |- (cu.bfunct7);
	\draw[] (dm.brd -| regdmread.bout) -- ++(1.1, 0) |- (rf.ba1);
	\draw[] (dm.brd -| regdmread.bout) -- ++(1.1, 0) |- (rf.ba2);
	\draw[] (dm.brd -| regdmread.bout) -- ++(1.1, 0) |- (rf.ba3);
	\draw[] (dm.brd -| regdmread.bout) -- ++(1.1, 0) |- (ext.bin);
	
	% extend unit
	\draw[] (ext.bout) -| ($(muxalusrcb.bin1) + (-0.75, 0)$) -- (muxalusrcb.bin1);
	
	% control unit
	\draw[custyle] (cu.bregwrite) -| (rf.bwe3);
	\draw[custyle] (cu.bimmsrc) -- ++(2.65, 0) |- ($(ext.bctrl) + (0, 0.5)$) -- (ext.bctrl);
	\draw[custyle] (cu.balusrca) -| (muxalusrca.bsel);
	\draw[custyle] (cu.balusrcb) -| (muxalusrcb.bsel);
	\draw[custyle] (cu.balucontrol) -| (alu.bctrl);
	\draw[custyle] (cu.bresultsrc) -| (muxresult.bsel);
	\draw[custyle] (regdmread.ben) -- ++(0, -0.25) -- ++(-0.4, 0) |- (cu.birwrite);
	\draw[custyle] (cu.bmemwrite) -| (dm.bwe);
	\draw[custyle] (cu.badrsrc) -| (muxaddrnext.bsel);
	\draw[custyle] (regpcnext.ben) -- ++(0, -0.25) -- ++(-0.4, 0) |- (cu.bpcwrite);
	
	% additional connections
	\draw[] (regpcnext.bout) -- (muxaddrnext.bin0);
	\draw[] (muxresult.bout) -- ++(0.25, 0) -- ++(0, -3.5) -| ($(regpcnext.bin) + (-0.5, 0)$) -- (regpcnext.bin);
	\draw[] (muxresult.bout) -- ++(0.25, 0) -- ++(0, -3.5) -| ($(muxaddrnext.bin1) + (-0.25, 0)$) -- (muxaddrnext.bin1);
	\draw[] (muxresult.bout) -- ++(0.25, 0) -- ++(0, -3.5) -| ($(rf.bwd3) + (-0.5, 0)$) -- (rf.bwd3);
	\draw[] ($(muxaddrnext.bin0) + (-0.25, 0)$)-- ++(0, 2.4) -| ($(muxalusrca.bin0) + (-0.25, 0)$) --(muxalusrca.bin0); % this is the connection from PC to the SrcA mux
	\draw[] ($(muxaddrnext.bin0) + (-0.25, 0)$) |- (regdmread.bin);
	\draw[] (regdmread.bout) -- ++(6.75, 0) -| ($(muxalusrca.bin1) + (-0.5, 0)$) -- (muxalusrca.bin1);
	\draw[] (regaluresult.bout) -- (muxresult.bin0);
	\draw[] (alu.bout) -- ++(1,0) |- (muxresult.bin2);
	
	% labels
	\node[align=right, label, left=0.25em of alu.ba, yshift=0.5em] {SrcA};
	\node[align=right, label, left=0.25em of alu.bb, yshift=0.5em] {SrcB};
	\node[align=left, label, right=0.25em of alu.bout, yshift=0.5em] {ALUResult};
	\node[align=left, label, right=0.25em of alu.bzero, yshift=0.5em] {Zero};
	\node[align=left, label, anchor=south west, xshift=0.3cm, rotate=-90] at (rf.brd2 -| dm.brd) {ReadData};
	\node[align=left, label, anchor=south west, xshift=0.5cm, rotate=-90] at (rf.brd2 -| regrfread.bout) {WriteData};
	\node[align=left, label, anchor=west, yshift=0.5em] at (dm.brd -| regdmread.bout) {Instr};
	\node[align=left, label, right=0.25em of ext.bout, yshift=0.5em] {ImmExt};
	
	\node[align=left, label, anchor=west, xshift=1.1cm, yshift=0.5em] at (regdmread.bout |- cu.bop) {\scriptsize{6:0}};
	\node[align=left, label, anchor=west, xshift=1.1cm, yshift=0.5em] at (regdmread.bout |- cu.bfunct3) {\scriptsize{14:12}};
	\node[align=left, label, anchor=west, xshift=1.1cm, yshift=0.5em] at (regdmread.bout |- cu.bfunct7) {\scriptsize{30}};
	\node[align=left, label, anchor=west, xshift=1.1cm, yshift=0.5em] at (regdmread.bout |- rf.ba1) {\scriptsize{19:15}};
	\node[align=left, label, anchor=west, xshift=1.1cm, yshift=0.5em] at (regdmread.bout |- rf.ba2) {\scriptsize{24:20}};
	\node[align=left, label, anchor=west, xshift=1.1cm, yshift=0.5em] at (regdmread.bout |- rf.ba3) {\scriptsize{11:7}};
	\node[align=left, label, anchor=west, xshift=1.1cm, yshift=0.5em] at (regdmread.bout |- ext.bin) {\scriptsize{31:7}};
	
	\node[align=left, label, custyle, right=0.25em of cu.bregwrite, yshift=0.5em] {RegWrite};
	\node[align=left, label, custyle, right=0.25em of cu.bimmsrc, yshift=0.5em] {ImmSrc\textsubscript{1:0}};
	\node[align=left, label, custyle, right=0.25em of cu.balusrca, yshift=0.5em] {ALUSrcA\textsubscript{1:0}};
	\node[align=left, label, custyle, right=0.25em of cu.balusrcb, yshift=0.5em] {ALUSrcB\textsubscript{1:0}};
	\node[align=left, label, custyle, right=0.25em of cu.balucontrol, yshift=0.5em] {ALUControl\textsubscript{2:0}};
	\node[align=left, label, custyle, right=0.25em of cu.bresultsrc, yshift=0.5em] {ResultSrc\textsubscript{1:0}};
	\node[align=right, label, custyle, left=0.25em of cu.birwrite, yshift=0.5em] {IRWrite};
	\node[align=right, label, custyle, left=0.25em of cu.bmemwrite, yshift=0.5em] {MemWrite};
	\node[align=right, label, custyle, left=0.25em of cu.badrsrc, yshift=0.5em] {AdrSrc};
	\node[align=right, label, custyle, left=0.25em of cu.bpcwrite, yshift=0.5em] {PCWrite};
	\node[align=right, label, left=0.25cm of regpcnext.bin, yshift=0.5em] {PCNext};
	\node[align=left, label, right=0.25em of muxaddrnext.bout, yshift=0.5em] {Adr};
	\node[align=left, label, right=0em of regpcnext.bout, yshift=0.5em] {PC};
	\node[align=left, label, right=0em of regdmread.bout, yshift=0.5em] {OldPC};
	\node[align=left, label, right=0em of regdmread2.bout, yshift=0.5em] {Data};
	\node[align=left, label, right=0em of regaluresult.bout, yshift=0.5em] {ALUOut};
	\node[align=left, label, left=0.5cm of muxalusrca.bin2, yshift=0.5em] {A};
	\coordinate (resultlabel) at ($(muxresult.bout) + (0.25, -3.5)$);
	\node[align=right, label, left=0.5em of resultlabel, yshift=0.5em] {Result};
	
\end{circuitikz}
	}
	\resizebox*{\textwidth}{!}{
		\def\bblue#1{\textbf{\textcolor{modblue}{#1}}}
\begin{tikzpicture}
	\tikzset{
		every node/.style={},
		line/.style={thick, line cap=round},
		arrow/.style={line, ->},
		mod/.style={font=\small, inner sep=-3pt, align=center, minimum size=3.35cm},
		label/.style={font=\ttfamily\small\linespread{0.5}\selectfont, align=center},
	}
	\node[state, mod] (s0) {\bblue{S0: Fetch}\\AdrSrc = 0\\IRWrite\\ALUSrcA = 00\\ALUSrcB = 10\\ALUOp = 00\\ResultSrc = 10\\PCUpdate};
	\node[state, mod, right=1cm of s0] (s1) {\bblue{S1: Decode}\\ALUSrcA = 01\\ALUSrcB = 01\\ALUOp = 00};
	\node[state, mod, below=1.5cm of s1, xshift=2.5cm] (s8) {\bblue{S8: ExecuteI}\\ALUSrcA = 10\\ALUSrcB = 01\\ALUOp = 10};
	\node[state, mod, left=1cm of s8] (s6) {\bblue{S6: ExecuteR}\\ALUSrcA = 10\\ALUSrcB = 00\\ALUOp = 10};
	\node[state, mod, right=1cm of s8] (s9) {\bblue{S9: JAL}\\ALUSrcA = 01\\ALUSrcB = 10\\ALUOp = 00\\ResultSrc = 00\\PCUpdate};
	\node[state, mod, below=1cm of s8] (s7) {\bblue{S7: ALUWB}\\ResultSrc = 00\\RegWrite};
	\node[state, mod, right=1cm of s9] (s10) {\bblue{S10: BEQ}\\ALUSrcA = 10\\ALUSrcB = 00\\ALUOp = 01\\ResultSrc = 00\\Branch};
	\node[state, mod, left=1cm of s6] (s2) {\bblue{S2: MemAddr}\\ALUSrcA = 10\\ALUSrcB = 01\\ALUOp = 00};
	\node[state, mod, below=1cm of s2] (s3) {\bblue{S3: MemRead}\\ResultSrc = 00\\AdrSrc = 1};
	\node[state, mod, right=1cm of s3] (s5) {\bblue{S5: MemWrite}\\ResultSrc = 00\\AdrSrc = 1\\MemWrite};
	\node[state, mod, below=1cm of s3] (s4) {\bblue{S4: MemWB}\\ResultSrc = 01\\RegWrite};
	
	\node[] (res) at ($(s0.north west) + (-.5, .5)$) {Reset};
	\draw[arrow] (res.south) -- (s0);
	
	\draw[arrow] (s0) -- (s1);
	\draw[arrow] (s1) -- (s2) node[left, pos=0.5, label] {op = 0000011 (lw) or\\op = 0100011 (sw)};
	\draw[arrow] (s1) -- (s6) node[left, pos=0.5, label] {op =\\0110011\\(R-Type)};
	\draw[arrow] (s1) -- (s8) node[right, pos=0.5, label] {op = \\0010011\\(I-Type ALU)};
	\draw[arrow] (s1) -- (s9) node[right, pos=0.5, label] {op = 1101111\\(jal)};
	\draw[arrow] (s1) -- (s10.north west) node[right, pos=0.5, label] {op = 1100011\\(beq)};
	
	\draw[arrow] (s2) -- (s3) node[left, pos=0.5, label] {op =\\0000011\\(lw)};
	\draw[arrow] (s2) -- (s5) node[left, pos=0.5, label] {op =\\0100011\\(sw)};
	\draw[arrow] (s3) -- (s4);
	
	\draw[arrow] (s6) -- (s7);
	\draw[arrow] (s8) -- (s7);
	\draw[arrow] (s9) -- (s7);
	
	\node[right=18.5 of s4] (anchorSE) {};
	\node[above=15.5cm of anchorSE] (anchorNE) {};
	
	\draw[line] (s4) -- (anchorSE.center);
	\draw[line] (s5) |- (anchorSE.center);
	\draw[line] (s7) |- (anchorSE.center);
	\draw[line] (s10) |- (anchorSE.center);
	\draw[line] (anchorSE.center) |- (anchorNE.center);
	\draw[arrow] (anchorNE.center) -| (s0.north);
	
\end{tikzpicture}
	}
\end{center}

\subsection{Pipelined RISC-V Processor}
\begin{center}
	\resizebox*{\textwidth}{!}{
		\begin{circuitikz}
	\tikzset{
		label/.style={draw=none, inner sep=1pt, font=\small},
		comp/.style={align=center, no leads},
		custyle/.style={color=modblue},
	}

	% components
	\node[regfile, comp] (rf) at (-4.5, 0) {Register\\File};
	\node[instrmem, comp, left=2.5cm of rf] (im) {Instruction\\Memory};
	\node[alu,comp, right=6cm of rf.rd1, anchor=a] (alu) {};
	\node[datamem,comp, right=11cm of rf] (dm) {Data\\Memory};

	\node[ctrlunitpl, comp, custyle, above=1cm of rf, xshift=-0.75cm] (cu) {Control\\Unit};

	\node[extend, comp, below=1.5cm of rf] (ext) {Extend};

	% TODO: align mux.out and alu.b
	\node[mux, comp, right=4.5cm of rf.rd2, anchor=in0] (muxalusrc) {};
	\node[reg, comp, left=0.3cm of im.a] (regpcnext) {};
	\node[mux, comp, left=1.25cm of regpcnext] (muxpcnext) {};
	\node[3mux, comp, right=1.75cm of dm.rd, anchor=in1] (muxresult) {};

	\node[adder, comp, anchor=b] (addpctarget) at (ext.out -| muxalusrc.bout) {};
	\node[adder, comp, anchor=a] (addpcplus4) at (ext.in -| im.a) {};

	% register
	\node[reg, comp, stacks=6.5, right=0.25cm of im.north east, anchor=north west] (regfetch) {};
	\node[reg, comp, no clock, stacks=6.5, right=2.75cm of rf.north east, anchor=north west] (regdecode) {};
	\node[reg, comp, no clock, stacks=6.5, right=8cm of rf.north east, anchor=north west] (regexecute) {};
	\node[reg, comp, no clock, stacks=8, right=4.5cm of regexecute.south east, anchor=south west] (regmem) {};

	\node[reg, comp, custyle, stacks=5, above=0cm of regdecode.north, anchor=south] (regdecodecu) {};
	\node[reg, comp, custyle, stacks=5, above=0cm of regexecute.north, anchor=south] (regexecutecu) {};
	\node[reg, comp, custyle, stacks=3.5, above=0cm of regmem.north, anchor=south] (regmemcu) {};

	%special nodes
	% clks
	\node[align=center, label, above=0.25cm of regpcnext.clk] (clk1) {CLK};
	\draw[] (clk1) -- (regpcnext.clk);
	\node[align=center, label, above=0.25cm of regfetch.clk] (clk2) {CLK};
	\draw[] (clk2) -- (regfetch.clk);
	\node[align=center, label, custyle, above=0.25cm of regdecodecu.clk] (clk3) {CLK};
	\draw[custyle] (clk3) -- (regdecodecu.clk);
	\node[align=center, label, custyle, above=0.25cm of regexecutecu.clk] (clk4) {CLK};
	\draw[custyle] (clk4) -- (regexecutecu.clk);
	\node[align=center, label, custyle, above=0.25cm of regmemcu.clk] (clk5) {CLK};
	\draw[custyle] (clk5) -- (regmemcu.clk);

	\node[align=center, label, above=0.25cm of rf.bclk] (clkrf) {CLK};
	\draw[] (clkrf) -- (rf.bclk);
	\node[align=center, label, above=0.25cm of dm.bclk] (clkdm) {CLK};
	\draw[] (clkdm) -- (dm.bclk);

	% constants
	\node[align=right, label, left=0.25cm of addpcplus4.bb] (const4) {4};
	\draw[] (const4) -- (addpcplus4.bb);

	% connections
	% register file
	\draw[] (rf.brd1) -- (rf.brd1 -| regdecode.bin);
	\draw[] (regdecode.bout |- alu.ba) -- (alu.ba);
	\draw[] (rf.brd2) -- (rf.brd2 -| regdecode.bin);
	\draw[] (regdecode.bout |- muxalusrc.bin0) -- (muxalusrc.bin0);
	\draw[] (muxalusrc.bin0) -- ++(-0.5, 0) |- (regexecute.bin |- dm.bwd);
	\draw[] (regexecute.bout |- dm.bwd) -- (dm.bwd);

	% alu
	\draw[] (muxalusrc.bout) |- (alu.bb);
	\draw[] (alu.bout) -| (regexecute.bin);
	\draw[] (regexecute.bout |- alu.bout) -- ++(2, 0) |- (dm.ba);

	\node[] (anchoraluresult) at ($(regexecute.bout |- alu.bout) + (2,2)$) {};
	\draw[] (regexecute.bout |- alu.bout) -| (anchoraluresult.center) -| (regmem.bin);
	\draw[] (anchoraluresult.center -| regmem.bout) -| ($(muxresult.bin0) + (-0.25, 0)$) -- (muxresult.bin0);

	% data memory
	\draw[] (dm.brd) -- (dm.brd-| regmem.bin);
	\draw[] (regmem.bout) |- (muxresult.bin1);
	\draw[] (muxresult.bout) -- ++(0.25,0) -- ++(0,-6) -| ($(rf.bwd3) + (-1.5, 0)$) node[align=center,above,pos=0.02] {ResultW} -- (rf.bwd3);

	% extend unit
	\draw[] (ext.bout) -- (ext.bout -| regdecode.bin);
	\draw[] (regdecode.bout |- addpctarget.bb) -- (addpctarget.bb);
	\draw[] (muxalusrc.bin1) -- ++(-0.25, 0)|- (addpctarget.bb);

	% instruction memory
	\draw[] (im.brd) -- (im.brd -| regfetch.bin);
	\draw[] (im.brd -| regfetch.bout) -- ++(1, 0) |- (cu.bop);
	\draw[] (im.brd -| regfetch.bout) -- ++(1, 0) |- (cu.bfunct3);
	\draw[] (im.brd -| regfetch.bout) -- ++(1, 0) |- (cu.bfunct7);
	\draw[] (im.brd -| regfetch.bout) -- ++(1, 0) |- (rf.ba1);
	\draw[] (im.brd -| regfetch.bout) -- ++(1, 0) |- (rf.ba2);
	\draw[] (im.brd -| regfetch.bout) -- ++(1, 0) |- (ext.bin);
	\draw[] (regpcnext.bout) -- node[align=center, label, above, pos=0.5] {PCF} (im.ba);

	\node[] (a3anchor) at ($(regfetch.bout) + (1, -2.75)$) {};
	\draw[] (im.brd -| regfetch.bout) -- ++(1, 0) -- (a3anchor.center);
	\draw[] (a3anchor.center) -- (a3anchor.center -| regdecode.bin);
	\draw[] (a3anchor.center -| regdecode.bout) -- (a3anchor.center -| regexecute.bin);
	\draw[] (a3anchor.center -| regexecute.bout) -- (a3anchor.center -| regmem.bin);
	\draw[] (a3anchor.center -| regmem.bout) -- ++(0.5, 0)  -- ++(0, -3.25)  -| ($(rf.ba3) + (-1.25, 0)$) -- (rf.ba3);

	% control unit
	\draw[custyle] (cu.bregwrite) -| (regdecodecu.bin);
	\draw[custyle] (cu.bresultsrc) -| (regdecodecu.bin);
	\draw[custyle] (cu.bmemwrite) -| (regdecodecu.bin);
	\draw[custyle] (cu.bjump) -| (regdecodecu.bin);
	\draw[custyle] (cu.bbranch) -| (regdecodecu.bin);
	\draw[custyle] (cu.balucontrol) -| (regdecodecu.bin);
	\draw[custyle] (cu.balusrc) -| (regdecodecu.bin);
	\draw[custyle] (cu.bimmsrc) -- ++(2,0) |- ($(ext.ctrl) + (0, 1.5)$) -- (ext.ctrl);

	\node[or port, custyle, right=2cm of regdecodecu.north east, yshift=1cm, scale=0.5, rotate=180] (or1) {};
	\node[and port, custyle, right=0cm of or1.in 2, anchor=bout, scale=0.5, rotate=180, no leads] (and1) {};

	\draw[custyle] (cu.bregwrite -| regdecodecu.bout) -| (regexecutecu.bin);
	\draw[custyle] (cu.bresultsrc -| regdecodecu.bout) -| (regexecutecu.bin);
	\draw[custyle] (cu.bmemwrite -| regdecodecu.bout) -| (regexecutecu.bin);
	\draw[custyle] (cu.bjump -| regdecodecu.bout) -| (or1.in 1);
	\draw[custyle] (cu.bbranch -| regdecodecu.bout) -| ($(and1.bin 1) + (0.125, 0)$)-- (and1.bin 1);
	\draw[custyle] (cu.balucontrol -| regdecodecu.bout) -| (alu.bctrl);
	\draw[custyle] (cu.balusrc -| regdecodecu.bout) -| (muxalusrc.bsel);
	\draw[custyle] (or1.out) -| (muxpcnext.bsel);
	\draw[] (alu.bzero) -- ++(0.125, 0) |- node[label, align=right,below,pos=0.6] {ZeroE} (and1.bin 2);

	\draw[custyle] (cu.bregwrite -| regexecutecu.bout) -| (regmemcu.bin);
	\draw[custyle] (cu.bresultsrc -| regexecutecu.bout) -| (regmemcu.bin);
	\draw[custyle] (cu.bmemwrite -| regexecutecu.bout) -| (dm.bwe);

	\draw[custyle] (cu.bregwrite -| regmemcu.bout) -- (cu.bregwrite -| muxresult.bsel) -- ++(0, 2) -| (rf.bwe3);
	\draw[custyle] (cu.bresultsrc -| regmemcu.bout) -| (muxresult.bsel);

	% additional connections
	\draw[] (addpcplus4.bout) -- ++(0.25,0) -- ++(0, -1.25)  -| node[align=right, label, above, pos=0.1] {PCPlus4F} ($(muxpcnext.bin0) + (-0.25, 0)$) -- (muxpcnext.bin0);
	\draw[] (addpcplus4.bout) -- (addpcplus4.bout -| regfetch.bin);
	\draw[] (addpcplus4.bout -| regfetch.bout) -- (addpcplus4.bout -| regdecode.bin);
	\draw[] (addpcplus4.bout -| regdecode.bout) -- (addpcplus4.bout -| regexecute.bin);
	\draw[] (addpcplus4.bout -| regexecute.bout) -- (addpcplus4.bout -| regmem.bin);
	\draw[] (addpcplus4.bout -| regmem.bout) -| ($(muxresult.bin2) + (-0.25, 0)$) -- (muxresult.bin2);
	\draw[] (regpcnext.bout) -- ++(0.3, 0) |- (addpcplus4.ba);
	\draw[] (regpcnext.bout) -- ++(0.3, 0) |- (regfetch.bin |- addpctarget.ba);
	\draw[] (regfetch.bout |- addpctarget.ba) -- (regdecode.bin |- addpctarget.ba);
	\draw[] (regdecode.bout |- addpctarget.ba) -- (addpctarget.ba);
	\draw[] (addpctarget.bout) -- ++(0.25, 0) -- ++(0, -2.75) -| node[align=right, label, above, pos=0.025] {PCTargetE}($(muxpcnext.bin1) + (-0.75, 0)$) -- (muxpcnext.bin1);
	\draw[] (muxpcnext.bout) -- (regpcnext.bin) node[align=center, label, above, pos=0.5] {PCF'};

	% labels
	% fetch
	% -

	% decode
	\node[label, align=left, anchor=west, yshift=0.5em] at (im.brd -| regfetch.bout) {InstrD};
	\node[label, custyle, align=left, right=0.5em of cu.bregwrite, yshift=0.5em] {RegWriteD};
	\node[label, custyle, align=left, right=0.5em of cu.bresultsrc, yshift=0.5em] {ResultSrcD\textsubscript{1:0}};
	\node[label, custyle, align=left, right=0.5em of cu.bmemwrite, yshift=0.5em] {MemWriteD};
	\node[label, custyle, align=left, right=0.5em of cu.bjump, yshift=0.5em] {JumpD};
	\node[label, custyle, align=left, right=0.5em of cu.bbranch, yshift=0.5em] {BranchD};
	\node[label, custyle, align=left, right=0.5em of cu.balucontrol, yshift=0.5em] {ALUControlD\textsubscript{2:0}};
	\node[label, custyle, align=left, right=0.5em of cu.balusrc, yshift=0.5em] {ALUSrcD};
	\node[label, custyle, align=left, right=0.5em of cu.bimmsrc, yshift=0.5em] {ImmSrcD\textsubscript{1:0}};

	\node[align=right, anchor=east, label, yshift=0.5em] at ($(a3anchor.center -| regdecode.bin) + (-0.25em, 0)$) {RdD};
	\node[align=right, anchor=east, label, yshift=0.5em] at ($(addpctarget.ba -| regdecode.bin) + (-0.25em, 0)$) {PCD};
	\node[align=right, anchor=east, label, yshift=0.5em] at ($(ext.bout-| regdecode.bin) + (-0.25em, 0)$) {ImmExtD};
	\node[align=right, anchor=east, label, yshift=0.5em] at ($(addpcplus4.bout-| regdecode.bin) + (-0.25em, 0)$) {PCPlus4D};

	\node[align=right, label, left=0.25em of cu.bop, yshift=0.5em] {\scriptsize{6:0}};
	\node[align=right, label, left=0.25em of cu.bfunct3, yshift=0.5em] {\scriptsize{14:12}};
	\node[align=right, label, left=0.25em of cu.bfunct7, yshift=0.5em] {\scriptsize{30}};
	\node[align=right, label, left=0.25em of rf.ba1, yshift=0.5em] {\scriptsize{19:15}};
	\node[align=right, label, left=0.25em of rf.ba2, yshift=0.5em] {\scriptsize{24:20}};
	\node[align=right, label, xshift=-0.25em, yshift=0.5em, anchor=east] at (ext.bin |- a3anchor.center) {\scriptsize{11:7}};
	\node[align=right, label, left=0.25em of ext.bin, yshift=0.5em] {\scriptsize{31:7}};

	% execute
	\node[align=right, label, left=0.25em of alu.ba, yshift=0.5em] {SrcAE};
	\node[align=right, label, left=0.25em of alu.bb, yshift=0.5em] {SrcBE};
	\node[label, custyle, align=left, anchor=west, yshift=0.5em] at ($(cu.bregwrite -| regdecodecu.bout) + (0.25em, 0)$) {RegWriteE};
	\node[label, custyle, align=left, anchor=west, yshift=0.5em] at ($(cu.bresultsrc -| regdecodecu.bout) + (0.25em, 0)$) {ResultSrcE\textsubscript{1:0}};
	\node[label, custyle, align=left, anchor=west, yshift=0.5em] at ($(cu.bmemwrite -| regdecodecu.bout) + (0.25em, 0)$) {MemWriteE};
	\node[label, custyle, align=left, anchor=west, yshift=0.5em] at ($(cu.bjump -| regdecodecu.bout) + (0.25em, 0)$) {JumpE};
	\node[label, custyle, align=left, anchor=west, yshift=0.5em] at ($(cu.bbranch -| regdecodecu.bout) + (0.25em, 0)$) {BranchE};
	\node[label, custyle, align=left, anchor=west, yshift=0.5em] at ($(cu.balucontrol -| regdecodecu.bout) + (0.25em, 0)$) {ALUControlE\textsubscript{2:0}};
	\node[label, custyle, align=left, anchor=west, yshift=0.5em] at ($(cu.balusrc -| regdecodecu.bout) + (0.25em, 0)$) {ALUSrcE};

	\node[align=left, anchor=west, label, yshift=0.5em] at ($(rf.brd1 -| regdecode.bout) + (0.25em, 0)$) {RD1E};
	\node[align=left, anchor=west, label, yshift=0.5em] at ($(rf.brd2 -| regdecode.bout) + (0.25em, 0)$) {RD2E};
	\node[align=left, anchor=west, label, yshift=0.5em] at ($(a3anchor.center -| regdecode.bout) + (0.25em, 0)$) {RdE};
	\node[align=left, anchor=west, label, yshift=0.5em] at ($(addpctarget.ba -| regdecode.bout) + (0.25em, 0)$) {PCE};
	\node[align=left, anchor=west, label, yshift=0.5em] at ($(ext.bout-| regdecode.bout) + (0.25em, 0)$) {ImmExtE};
	\node[align=left, anchor=west, label, yshift=0.5em] at ($(addpcplus4.bout-| regdecode.bout) + (0.25em, 0)$) {PCPlus4E};
	\node[label, custyle, align=right, anchor=east, yshift=0.5em] at ($(or1.out) + (0.25em, 0)$) {PCSrcE};

	% memory
	\node[label, custyle, align=left, anchor=west, yshift=0.5em] at ($(cu.bregwrite -| regexecutecu.bout) + (0.25em, 0)$) {RegWriteM};
	\node[label, custyle, align=left, anchor=west, yshift=0.5em] at ($(cu.bresultsrc -| regexecutecu.bout) + (0.25em, 0)$) {ResultSrcM\textsubscript{1:0}};
	\node[label, custyle, align=left, anchor=west, yshift=0.5em] at ($(cu.bmemwrite -| regexecutecu.bout) + (0.25em, 0)$) {MemWriteM};


	\node[align=left, anchor=west, label, yshift=0.5em] at ($(alu.bout -| regexecute.bout) + (0.25em, 0)$) {ALUResultM};
	\node[align=left, anchor=west, label, yshift=0.5em] at ($(dm.bwd -| regexecute.bout) + (0.25em, 0)$) {WriteDataM};
	\node[align=left, anchor=west, label, yshift=0.5em] at ($(a3anchor.center -| regexecute.bout) + (0.25em, 0)$) {RdM};
	\node[align=left, anchor=west, label, yshift=0.5em] at ($(addpcplus4.bout-| regexecute.bout) + (0.25em, 0)$) {PCPlus4D};

	% writeback
	\node[label, custyle, align=left, anchor=west, yshift=0.5em] at ($(cu.bregwrite -| regmemcu.bout) + (0.25em, 0)$) {RegWriteW};
	\node[label, custyle, align=left, anchor=west, yshift=0.5em] at ($(cu.bresultsrc -| regmemcu.bout) + (0.25em, 0)$) {ResultSrcW\textsubscript{1:0}};

	\node[align=left, anchor=west, label, yshift=0.5em] at ($(dm.brd -| regmem.bout) + (0.25em, 0)$) {ReadDataW};
	\node[align=left, anchor=west, label, yshift=0.5em] at ($(a3anchor.center -| regmem.bout) + (0.25em, 0)$) {RdW};
	\node[align=left, anchor=west, label, yshift=0.5em] at ($(addpcplus4.bout-| regmem.bout) + (0.25em, 0)$) {PCPlus4W};
\end{circuitikz}
	}
\end{center}

\end{document}

\end{document}

