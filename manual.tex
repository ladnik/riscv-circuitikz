\documentclass[12pt,a4paper,titlepage]{article}

\usepackage{a4wide}
\usepackage[T1]{fontenc}
\usepackage{tikz}
\usepackage{circuitikz}
\usetikzlibrary{calc, riscvproc, positioning, fit, decorations, decorations.pathmorphing}
\usepackage{ctikzmanutils}
\usepackage{ifthen}
\usepackage{xparse}
\usepackage{showexpl}
\usepackage{ragged2e}

\parindent=0pt
\parskip=4pt plus 6pt minus 2pt

%
% modified gelrcoord: https://github.com/circuitikz/circuitikz/blob/master/doc/ctikzmanutils.sty
%
\newcommand{\modgeolrcoord}[2][]{\showanchors[#1]{#2}{text}(north/90/0.4, north east/45/0.4, east/0/0.4,
	south east/-45/0.4,
	south/-90/0.4, south west/-135/0.4, west/180/0.4, north west/135/0.4,
	left/160/0.4, right/30/0.4, center/-120/0.3
	)
}
\newcommand\defaultanchors{
}

\title{RISC-V Processor \Circuitikz{} Library}
\date{\today}

\begin{document}

\begin{center}
	\LARGE \textbf{\thetitle}

	\normalsize \thedate
\end{center}

%\circuitdesc*{instrmem}{Instruction Memory}{}()
%\modgeolrcoord{instrmem}
%\showanchors{instrmem, align=center}{Instruction\\Memory}(
%a/180/0.4,
%rd/0/0.4
%)

\section{Components}

\subsection{Instruction Memory}
\begin{center}
	
	\begin{circuitikz}[]
		\node[instrmem, align=center] (comp) {Instruction\\Memory};

		\draw[->, red] (comp.a) -- ++(-1, 0) node[left] {a};
		\draw[->, blue] (comp.rd) -- ++(1, 0) node[right] {rd};
	\end{circuitikz}
\end{center}

\subsection{Data Memory}
\begin{center}
	\begin{circuitikz}[]
		\node[datamem, align=center] (comp) {Data\\Memory};
		\draw[->, red] (comp.a) -- ++(-1, 0) node[left] {a};
		\draw[->, red] (comp.wd) -- ++(-1, 0) node[left] {wd};
		\draw[->, red] (comp.clk) -- ++(0, 1) node[above] {clk};
		\draw[->, red] (comp.we) -- ++(0, 1) node[above] {we};
		\draw[->, blue] (comp.rd) -- ++(1, 0) node[right] {rd};
	\end{circuitikz}
\end{center}

\subsection{Register File}
\begin{center}
	\begin{circuitikz}[]
		\node[regfile, align=center] (comp) {Register\\File};
		\draw[->, red] (comp.a1) -- ++(-1, 0) node[left] {a1};
		\draw[->, red] (comp.a2) -- ++(-1, 0) node[left] {a2};
		\draw[->, red] (comp.a3) -- ++(-1, 0) node[left] {a3};
		\draw[->, red] (comp.wd3) -- ++(-1, 0) node[left] {wd3};
		
		\draw[->, red] (comp.clk) -- ++(0, 1) node[above] {clk};
		\draw[->, red] (comp.we3) -- ++(0, 1) node[above] {we3};
		\draw[->, blue] (comp.rd1) -- ++(1, 0) node[right] {rd1};
		\draw[->, blue] (comp.rd2) -- ++(1, 0) node[right] {rd2};
	\end{circuitikz}
\end{center}

\subsection{Extend Unit}
\begin{center}
	\begin{circuitikz}[]
		\node[extend, align=center] (comp) {Extend};
		\draw[->, red] (comp.in) -- ++(-1, 0) node[left] {in};
		\draw[->, blue] (comp.out) -- ++(1, 0) node[right] {out};
	\end{circuitikz}
\end{center}

\subsection{Arithmetic Logic Unit}
\begin{center}
	\begin{circuitikz}[]
		\node[alu, align=center] (comp) {ALU};
		\draw[->, red] (comp.a) -- ++(-1, 0) node[left] {a};
		\draw[->, red] (comp.b) -- ++(-1, 0) node[left] {b};
		\draw[->, red] (comp.ctrl) -- ++(0, 1) node[above] {ctrl};
		\draw[->, blue] (comp.out) -- ++(1, 0) node[right] {out};
		\draw[->, blue] (comp.zero) -- ++(1, 0) node[right] {zero};
	\end{circuitikz}
\end{center}

\subsection{Register}
\begin{center}
	\begin{circuitikz}[]
		\node[reg, align=center] (comp) {};
		\draw[->, red] (comp.in) -- ++(-1, 0) node[left] {in};
		\draw[->, red] (comp.clk) -- ++(0, 1) node[above] {clk};
		\draw[->, blue] (comp.out) -- ++(1, 0) node[right] {out};
	\end{circuitikz}

\subsection{Adder}
\begin{center}
	\begin{circuitikz}[]
		\node[adder, align=center] (comp) {+};
		\draw[->, red] (comp.a) -- ++(-1, 0) node[left] {a};
		\draw[->, red] (comp.b) -- ++(-1, 0) node[left] {b};
		\draw[->, blue] (comp.out) -- ++(1, 0) node[right] {out};
	\end{circuitikz}
\end{center}
	
\end{center}
\end{document}

