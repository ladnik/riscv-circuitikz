\documentclass[.52pt,a4paper,titlepage]{article}

\usepackage{a4wide}
\usepackage[T1]{fontenc}
\usepackage{tikz}
\usepackage{circuitikz}
\usetikzlibrary{calc, riscvproc, positioning, fit, decorations, decorations.pathmorphing}
\usepackage{ctikzmanutils}
\usepackage{ifthen}
\usepackage{xparse}
\usepackage{showexpl}
\usepackage{ragged2e}

\parindent=0pt
\parskip=4pt plus 6pt minus 2pt

%
% modified gelrcoord: https://github.com/circuitikz/circuitikz/blob/master/doc/ctikzmanutils.sty
%
\newcommand{\modgeolrcoord}[2][]{\showanchors[#1]{#2}{text}(north/90/0.4, north east/45/0.4, east/0/0.4,
	south east/-45/0.4,
	south/-90/0.4, south west/-.535/0.4, west/.580/0.4, north west/.535/0.4,
	left/.560/0.4, right/30/0.4, center/-.520/0.3
	)
}
\newcommand\defaultanchors{
}

\title{RISC-V Processor \Circuitikz{} Library}
\date{\today}

\begin{document}

\begin{center}
	\LARGE \textbf{\thetitle}

	\normalsize \thedate
\end{center}

%\circuitdesc*{instrmem}{Instruction Memory}{}()
%\modgeolrcoord{instrmem}
%\showanchors{instrmem, align=center}{Instruction\\Memory}(
%a/.580/0.4,
%rd/0/0.4
%)

\section{Components}

\subsection{Instruction Memory}
\begin{center}
	\begin{LTXexample}[varwidth, rframe=]
		\begin{circuitikz}
			\node[instrmem, align=center] (comp) {Instruction\\Memory};

			\draw[->, red] (comp.a) -- ++(-.5, 0) node[left] {a};
			\draw[->, blue] (comp.rd) -- ++(.5, 0) node[right] {rd};
		\end{circuitikz}
	\end{LTXexample}
\end{center}



\subsection{Data Memory}
\begin{center}
	\begin{LTXexample}[varwidth, rframe=]
		\begin{circuitikz}[]
			\node[datamem, align=center] (comp) {Data\\Memory};
			\draw[->, red] (comp.a) -- ++(-.5, 0) node[left] {a};
			\draw[->, red] (comp.wd) -- ++(-.5, 0) node[left] {wd};
			\draw[->, red] (comp.clk) -- ++(0, .5) node[above] {clk};
			\draw[->, red] (comp.we) -- ++(0, .5) node[above] {we};
			\draw[->, blue] (comp.rd) -- ++(.5, 0) node[right] {rd};
		\end{circuitikz}
	\end{LTXexample}
\end{center}

\subsection{Register File}
\begin{center}
	\begin{LTXexample}[varwidth, rframe=]
		\begin{circuitikz}[]
			\node[regfile, align=center] (comp) {Register\\File};
			\draw[->, red] (comp.a1) -- ++(-.5, 0) node[left] {a1};
			\draw[->, red] (comp.a2) -- ++(-.5, 0) node[left] {a2};
			\draw[->, red] (comp.a3) -- ++(-.5, 0) node[left] {a3};
			\draw[->, red] (comp.wd3) -- ++(-.5, 0) node[left] {wd3};

			\draw[->, red] (comp.clk) -- ++(0, .5) node[above] {clk};
			\draw[->, red] (comp.we3) -- ++(0, .5) node[above] {we3};
			\draw[->, blue] (comp.rd1) -- ++(.5, 0) node[right] {rd1};
			\draw[->, blue] (comp.rd2) -- ++(.5, 0) node[right] {rd2};
		\end{circuitikz}
	\end{LTXexample}
\end{center}

\subsection{Extend Unit}
\begin{center}
	\begin{LTXexample}[varwidth, rframe=]
		\begin{circuitikz}[]
			\node[extend, align=center] (comp) {Extend};
			\draw[->, red] (comp.in) -- ++(-.5, 0) node[left] {in};
			\draw[->, blue] (comp.out) -- ++(.5, 0) node[right] {out};
		\end{circuitikz}
	\end{LTXexample}
\end{center}

\subsection{Arithmetic Logic Unit}
\begin{center}
	\begin{LTXexample}[varwidth, rframe=]
		\begin{circuitikz}[]
			\node[alu, align=center] (comp) {ALU};
			\draw[->, red] (comp.a) -- ++(-.5, 0) node[left] {a};
			\draw[->, red] (comp.b) -- ++(-.5, 0) node[left] {b};
			\draw[->, red] (comp.ctrl) -- ++(0, .5) node[above] {ctrl};
			\draw[->, blue] (comp.out) -- ++(.5, 0) node[right] {out};
			\draw[->, blue] (comp.zero) -- ++(.5, 0) node[right] {zero};
		\end{circuitikz}
	\end{LTXexample}
\end{center}

\subsection{Register}
\begin{center}
	\begin{LTXexample}[varwidth, rframe=]
		\begin{circuitikz}[]
			\node[reg, align=center] (comp) {};
			\draw[->, red] (comp.in) -- ++(-.5, 0) node[left] {in};
			\draw[->, red] (comp.clk) -- ++(0, .5) node[above] {clk};
			\draw[->, blue] (comp.out) -- ++(.5, 0) node[right] {out};
		\end{circuitikz}
	\end{LTXexample}
\end{center}
\subsection{Adder}
\begin{center}
	\begin{LTXexample}[varwidth, rframe=]
		\begin{circuitikz}[]
			\node[adder, align=center] (comp) {\textbf{+}};
			\draw[->, red] (comp.a) -- ++(-.5, 0) node[left] {a};
			\draw[->, red] (comp.b) -- ++(-.5, 0) node[left] {b};
			\draw[->, blue] (comp.out) -- ++(.5, 0) node[right] {out};
		\end{circuitikz}
	\end{LTXexample}
\end{center}

\subsection{Multiplexer}
\begin{center}
	\begin{LTXexample}[varwidth, rframe=]
		\begin{circuitikz}[]
			\node[mux, align=center] (comp) {};
			\draw[->, red] (comp.in1) -- ++(-.5, 0) node[left] {in1};
			\draw[->, red] (comp.in2) -- ++(-.5, 0) node[left] {in2};
			\draw[->, red] (comp.sel) -- ++(0, .5) node[above] {sel};
			\draw[->, blue] (comp.out) -- ++(.5, 0) node[right] {out};
		\end{circuitikz}
	\end{LTXexample}
\end{center}

\subsection{Single-Cycle Control Unit}
\begin{center}
	\begin{LTXexample}[varwidth, rframe=]
		\begin{circuitikz}[]
			\node[ctrlunitsc, align=center] (comp) {Control\\Unit};
			\draw[->, red] (comp.op) -- ++(-.5, 0) node[left] {op};
			\draw[->, red] (comp.funct3) -- ++(-.5, 0) node[left] {funct3};
			\draw[->, red] (comp.funct7) -- ++(-.5, 0) node[left] {funct7};
			\draw[->, red] (comp.zero) -- ++(-.5, 0) node[left] {zero};

			\draw[->, blue] (comp.pcsrc) -- ++(.5, 0) node[right] {pcsrc};
			\draw[->, blue] (comp.resultsrc) -- ++(.5, 0) node[right] {resultsrc};
			\draw[->, blue] (comp.memwrite) -- ++(.5, 0) node[right] {memwrite};
			\draw[->, blue] (comp.alucontrol) -- ++(.5, 0) node[right] {alucontrol};
			\draw[->, blue] (comp.alusrc) -- ++(.5, 0) node[right] {alusrc};
			\draw[->, blue] (comp.immsrc) -- ++(.5, 0) node[right] {immsrc};
			\draw[->, blue] (comp.regwrite) -- ++(.5, 0) node[right] {regwrite};
		\end{circuitikz}
	\end{LTXexample}
\end{center}


\end{document}

