\documentclass[.52pt,a4paper,titlepage]{article}

\usepackage{a4wide}
\usepackage[T1]{fontenc}
\usepackage{hyperref}
\usepackage{tikz}
\usepackage{circuitikz}
\usepackage{ctikzmanutils}
\usetikzlibrary{calc, positioning, fit, decorations, decorations.pathmorphing, riscvproc}

\usepackage{showexpl}
\usepackage{fancyhdr} 
\usepackage[dvipsnames]{xcolor}
\usepackage{lmodern}
\usepackage{tabularx}
\usepackage{caption}

\renewcommand{\rmdefault}{lmss} % set font to sans serif

\hypersetup{
	pdfauthor={Niklas Ladurner},
	pdftitle={RISC-V Processor \Circuitikz{} Library},
	pdfsubject={},
	pdfkeywords={riscv, processor, circuitikz},
	pdfcreator={pdflatex}
}

\parindent=0pt
\parskip=4pt plus 6pt minus 2pt

\lstset{
	gobble=6,
	basicstyle=\ttfamily\small
}

\definecolor{modblue}{HTML}{288DCC}

% https://github.com/circuitikz/circuitikz/blob/c94be92df781cfbcba21c6d9c5e95a98976917a5/doc/ctikzmanutils.sty#L108
\newcommand{\widetwopartbox}[2]{%
	\leavevmode\null\par\noindent\fbox{\parbox[c]{0.4\linewidth}{#1} \parbox[c]{0.6\linewidth}{\RaggedRight\hbadness=9500 #2}\par\noindent}%
}
\NewDocumentCommand{\widecircuitdesc}{s O{1} m m m d() d[]}
{
	\widetwopartbox{%
		\centering
		\begin{circuitikz}[]
			\IfBooleanTF{#1}{%
				\draw (0,0) node[#3,scale=#2, fill=fillcol](N){#5};
			}{
				% if it's non-fillable, red should not go through
				\draw (0,0) node[#3,scale=#2,
				% fill=red
				](N){#5};
			}
			\IfValueT{#6}{%
				\foreach \n/\a/\d in {#6} \path(N.\n) \showcoord(\n)<\a:\d>;
			}
			\IfValueT{#7}{%
				\foreach \n/\a/\d in {#7} \path(N-\n) \showcoordb(N-\n)<\a:\d>;
			}
		\end{circuitikz}%
	}{\sloppy
		{#4, type: node\IfBooleanT{#1}{, fillable}%
		} (\texttt{node[\detokenize{#3}]\IfValueT{#7}{(N)}\{\detokenize{#5}\}}). \index{#3}%
		\checkclass{N}%
	}%
}


\title{A RISC-V Processor Components \Circuitikz{} Library}
\author{Niklas Ladurner}
\date{\today}

\begin{document}

\begin{center}
	\LARGE \textbf{\thetitle}

	\normalsize \thedate
\end{center}

\tableofcontents

\section{Introduction}
\subsection{Motivation}
\subsection{Usage}

\newpage
\section{Component List}
\subsection{Memory Components}
\widecircuitdesc*{instrmem}{Instruction memory}{}(a/180/0.2, rd/0/0.2, ba/135/0.2, brd/45/0.2)

\widecircuitdesc*{datamem}{Data Memory}{}(a/180/0.2, wd/180/0.2, rd/0/0.2, clk/90/0.2,we/90/0.2, ba/135/0.2, bwd/225/0.2, brd/45/0.2, bclk/135/0.2,bwe/45/0.2)

\widecircuitdesc*{regfile}{Register File}{}(
a1/180/0.2,ba1/135/0.2,
a2/180/0.2,ba2/135/0.05,
a3/180/0.2,ba3/225/0.05,
wd3/180/0.2,bwd3/225/0.2,
rd1/0/0.2,brd1/45/0.2,
rd2/0/0.2,brd2/-45/0.2,
clk/90/0.2,bclk/135/0.2,
we3/90/0.2,bwe3/45/0.2)

\subsection{Miscellaneous Components}
\widecircuitdesc*{extend}{Extend Unit}{}(
in/180/0.2,bin/135/0.2,
out/0/0.2,bout/45/0.2,
ctrl/90/0.2,bctrl/135/0.2)

\widecircuitdesc*{reg}{Register}{}(
in/180/0.2,bin/135/0.2,
out/0/0.2,bout/45/0.2,
clk/90/0.2,bclk/135/0.2)

\subsection{Arithmetic Components}
\widecircuitdesc*{alu}{Arithmetic Logic Unit}{ALU}(
a/180/0.2,ba/135/0.2,
b/180/0.2,bb/225/0.2,
out/0/0.2,bout/-45/0.2,
zero/0/0.2,bzero/45/0.2,
ctrl/90/0.2,bctrl/45/0.2)


\widecircuitdesc*{adder}{Adder}{}(
a/180/0.2,ba/135/0.2,
b/180/0.2,bb/225/0.2,
out/0/0.2,bout/45/0.2)


\widecircuitdesc*{subtr}{Subtractor}{}(
a/180/0.2,ba/135/0.2,
b/180/0.2,bb/225/0.2,
out/0/0.2,bout/45/0.2)

\subsection{Multiplexers}
\widecircuitdesc*{mux}{Multiplexer}{}(
in0/180/0.2,bin0/135/0.2,
in1/180/0.2,bin1/225/0.2,
sel/90/0.2,bsel/45/0.2,
out/0/0.2,bout/45/0.2)

\widecircuitdesc*{3mux}{Multiplexer with 3 inputs}{}(
in0/180/0.2,bin0/135/0.2,
in1/180/0.2,bin1/225/0.025,
in2/180/0.2,bin2/225/0.2,
sel/90/0.2,bsel/45/0.2,
out/0/0.2,bout/45/0.2)

\subsection{Control Units}
\widecircuitdesc*{ctrlunitsc}{Single-Cycle Control Unit}{}(
op/180/0.5,bop/135/0.025,
funct3/180/0.5,bfunct3/135/0.025,
funct7/180/0.5,bfunct7/135/0.025,
zero/180/0.5,bzero/135/0.025,
pcsrc/0/0.5,bpcsrc/45/0.025,
resultsrc/0/0.5,bresultsrc/45/0.025,
memwrite/0/0.5,bmemwrite/45/0.025,
alucontrol/0/0.5,balucontrol/45/0.025,
alusrc/0/0.5,balusrc/45/0.025,
immsrc/0/0.5,bimmsrc/45/0.025,
regwrite/0/0.5,bregwrite/45/0.025)

\widecircuitdesc*{ctrlunitmc}{Multi-Cycle Control Unit}{}(
clk/90/0.2,bclk/135/0.2,
op/180/0.5,bop/135/0.025,
funct3/180/0.5,bfunct3/135/0.025,
funct7/180/0.5,bfunct7/135/0.025,
zero/180/0.5,bzero/135/0.025,
pcwrite/180/0.5,bpcwrite/135/0.025,
pcwrite/180/0.5,bpcwrite/135/0.025,
adrsrc/180/0.5,badrsrc/135/0.025,
memwrite/180/0.5,bmemwrite/135/0.025,
irwrite/180/0.5,birwrite/135/0.025,
resultsrc/0/0.5,bresultsrc/45/0.025,
alucontrol/0/0.5,balucontrol/45/0.025,
alusrca/0/0.5,balusrca/45/0.025,
alusrcb/0/0.5,balusrcb/45/0.025,
immsrc/0/0.5,bimmsrc/45/0.025,
regwrite/0/0.5,bregwrite/45/0.025)


\section{Keys}
\subsection{\Circuitikz{} keys}
The desired \Circuitikz{} key can be set via \verb|\ctikzset{processor/<key>=value}|. E.g. if one whishes to set the line width of all components to 4, the line  \verb|\ctikzset{processor/thickness=4}| would have to be included in the specific circuitikz picture. A list of all \Circuitikz{} keys can be found in Table \ref{tab:keys}. A list of component families can be found in Table \ref{tab:families}.

\begin{figure}[htbp]
	\begin{tabularx}{\textwidth}{|lXr|}
		\hline
		Key                       & Description                                                           & Default value    \\
		\hline
		\texttt{scale}            & Sets scale for all processor components.                              & \texttt{1}       \\
		\texttt{thickness}        & Sets line width for all processor components.                         & \texttt{2}       \\
		\texttt{font}             & Sets font family for all labels of processor components.              & \verb|\rmfamily| \\
		\texttt{memory/height}    & Sets height for all memory components.                                & \texttt{2}       \\
		\texttt{memory/width}     & Sets width for all memory components except \texttt{regfile}.         & \texttt{1.25}    \\
		\texttt{control/heightsc} & Sets height for \texttt{ctrlunitsc}.                                  & \texttt{2.5}     \\
		\texttt{control/heightmc} & Sets height for \texttt{ctrlunitmc}.                                  & \texttt{3.5}     \\
		\texttt{control/width}    & Sets width for control components.                                    & \texttt{0.9}     \\
		\texttt{control/radius}   & Sets border radius for control components.                            & \texttt{5}       \\
		\texttt{arith/height}     & Sets height for arithmetic components.                                & \texttt{0.9}     \\
		\texttt{arith/width}      & Sets height for arithmetic components.                                & \texttt{0.7}     \\
		\texttt{arith/slope}      & Sets slope for arithmetic components in degrees.                      & \texttt{15}      \\
		\texttt{extend/height}    & Sets height for big side of extend components.                        & \texttt{0.6}     \\
		\texttt{extend/width}     & Sets height for extend components.                                    & \texttt{2}       \\
		\texttt{extend/slope}     & Sets slope for extend components in degrees.                          & \texttt{7}       \\
		\texttt{mux/slope}        & Sets slope for multiplexers in degrees.                               & \texttt{15}      \\
		\texttt{misc/smallheight} & Sets height for small components.                                     & \texttt{0.65}    \\
		\texttt{misc/smallwidth}  & Sets width for small components. Also affects the CLK input triangle. & \texttt{0.3}     \\
		\texttt{misc/leadlen}     & Sets length for input and output leads.                               & \texttt{0.25}    \\
		\hline
	\end{tabularx}
	\label{tab:keys}
	\captionof{table}{List of \Circuitikz{} keys}
\end{figure}

\begin{figure}[htbp]
	\begin{tabularx}{\textwidth}{|lX|}
		\hline
		Component family      & Component list                                        \\
		\hline
		memory components     & \texttt{instrmem}, \texttt{datamem}, \texttt{regfile} \\
		control components    & \texttt{ctrlunitsc}, \texttt{ctrlunitmc}              \\
		arithmetic components & \texttt{alu}, \texttt{add}, \texttt{sub}              \\
		extend components     & \texttt{extend}                                       \\
		small components      & \texttt{mux}, \texttt{reg}                            \\
		\hline
	\end{tabularx}
	\label{tab:families}
	\captionof{table}{List of component families}
\end{figure}
\subsection{Special node keys}

\newpage
\section{Examples}
\subsection{Single-Cycle RISC-V Processor}
\begin{center}
	\resizebox*{\textwidth}{!}{
		\begin{circuitikz}
			\tikzset{
				label/.style={draw=none, inner sep=1pt, font=\small},
				comp/.style={align=center, no leads},
				custyle/.style={color=modblue},
			}

			% components
			\node[regfile, comp] (rf) {Register\\File};
			\node[instrmem, comp, left=2cm of rf] (im) {Instruction\\Memory};
			\node[alu,comp, right=2.25cm of rf.rd1, anchor=a] (alu) {ALU};
			\node[datamem,comp, right=6.25cm of rf] (dm) {Data\\Memory};

			\node[ctrlunitsc, comp, custyle, above=1cm of rf, xshift=-0.75cm] (cu) {Control\\Unit};

			\node[extend, comp, below=1cm of rf] (ext) {Extend};
			\node[adder, comp, anchor=b] (addpctarget) at (ext.out -| alu.a) {};
			\node[adder, comp, anchor=a] (addpcplus4) at (ext.in -| im.a) {};

			% TODO: align mux.out and alu.b
			\node[mux, comp, right=0.9cm of rf.rd2, anchor=in0] (muxalusrc) {};
			\node[reg, comp, left=0.3cm of im.a] (regpcnext) {};
			\node[mux, comp, left=1.25cm of regpcnext] (muxpcnext) {};
			\node[mux, comp, right=1cm of dm.rd, anchor=in1] (muxresult) {};

			% special nodes
			% clks
			\node[align=center, label, above=0.25cm of regpcnext.clk] (clk1) {CLK};
			\draw[] (clk1) -- (regpcnext.clk);
			\node[align=center, label, above=0.25cm of rf.bclk] (clkrf) {CLK};
			\draw[] (clkrf) -- (rf.bclk);
			\node[align=center, label, above=0.25cm of dm.bclk] (clkdm) {CLK};
			\draw[] (clkdm) -- (dm.bclk);

			% constants
			\node[align=right, label, left=0.25cm of addpcplus4.bb] (const4) {4};
			\draw[] (const4) -- (addpcplus4.bb);

			% connections
			% register file
			\draw[] (rf.brd1) -- (alu.ba);
			\draw[] (rf.brd2) -- (muxalusrc.bin0);
			\draw[] (rf.brd2) -- ++(1, 0) |- (dm.bwd);

			% alu
			\draw[] (muxalusrc.bout) |- (alu.bb);
			\draw[] (alu.bout) -| (dm.ba);
			\draw[] (alu.bzero) -- ++(1, 0) -- ++(0, 1.5)-- ++(-8,0)  |- (cu.bzero) ;

			% data memory
			\draw[] (dm.brd) -- (muxresult.bin1);
			\draw[] (alu.bout) -- ++(2,0) -- ++(0, 2) -- ++(3.5,0) |- (muxresult.bin0);
			\draw[] (muxresult.bout) -- ++(0.25,0) -- ++(0,-5) -- ++(-14.3, 0) node[align=center,above,pos=0.05] {Result} |- (rf.bwd3);

			% extend unit
			\draw[] (ext.bout) -- (addpctarget.bb);
			\draw[] (ext.bout) ++(1.5, 0)|- (muxalusrc.bin1);

			% instruction memory
			\draw[] (im.brd) -- ++(1.15, 0) |- (cu.bop);
			\draw[] (im.brd) -- ++(1.15, 0) |- (cu.bfunct3);
			\draw[] (im.brd) -- ++(1.15, 0) |- (cu.bfunct7);
			\draw[] (im.brd) -- ++(1.15, 0) |- (rf.ba1);
			\draw[] (im.brd) -- ++(1.15, 0) |- (rf.ba2);
			\draw[] (im.brd) -- ++(1.15, 0) |- (rf.ba3);
			\draw[] (im.brd) -- ++(1.15, 0) |- (ext.bin);
			\draw[] (regpcnext.bout) -- (im.ba);

			% control unit
			\draw[custyle] (cu.bregwrite) -| (rf.bwe3);
			\draw[custyle] (cu.bimmsrc) -- ++(2, 0) -- ++(0, -5.5) -| (ext.bctrl);
			\draw[custyle] (cu.balusrc) -| (muxalusrc.bsel);
			\draw[custyle] (cu.balucontrol) -| (alu.bctrl);
			\draw[custyle] (cu.bmemwrite) -| (dm.bwe);
			\draw[custyle] (cu.bresultsrc) -| (muxresult.bsel);
			\draw[custyle] (cu.bpcsrc) -- ++(1.25, 0) -- ++(0, 0.75)-| (muxpcnext.bsel);

			% additional connections
			\draw[] (addpcplus4.bout) -- ++(0.25,0) -- ++(0, -1.25)  -- ++ (-4.5,0) |- (muxpcnext.bin0);
			\draw[] (regpcnext.bout) -- ++(0.3, 0) |- (addpcplus4.ba);
			\draw[] (regpcnext.bout) -- ++(0.3, 0) |- (addpctarget.ba);
			\draw[] (addpctarget.bout) -- ++(0.25, 0) -- ++(0, -2.75) -- ++ (-14.5,0) |- (muxpcnext.bin1);
			\draw[] (muxpcnext.bout) -- (regpcnext.bin) node[align=center, label, above, pos=0.5] {PCNext};

			% labels
			\node[align=right, label, left=0.25em of alu.ba, yshift=0.5em] {SrcA};
			\node[align=right, label, left=0.25em of alu.bb, yshift=0.5em] {SrcB};
			\node[align=left, label, right=0.25em of alu.bout, yshift=0.5em] {ALUResult};
			\node[align=left, label, right=0.25em of alu.bzero, yshift=0.5em] {Zero};
			\node[align=left, label, right=0.25em of dm.brd, yshift=0.5em] {ReadData};
			\node[align=left, label, right=0.25em of alu.bout, yshift=-1.15cm+0.5em] {WriteData};
			\node[align=left, label, right=0.25em of ext.bout, yshift=0.5em] {ImmExt};
			\node[align=right, label, left=0.25em of cu.bop, yshift=0.5em] {\scriptsize{6:0}};
			\node[align=right, label, left=0.25em of cu.bfunct3, yshift=0.5em] {\scriptsize{14:12}};
			\node[align=right, label, left=0.25em of cu.bfunct7, yshift=0.5em] {\scriptsize{30}};
			\node[align=right, label, left=0.25em of rf.ba1, yshift=0.5em] {\scriptsize{19:15}};
			\node[align=right, label, left=0.25em of rf.ba2, yshift=0.5em] {\scriptsize{24:20}};
			\node[align=right, label, left=0.25em of rf.ba3, yshift=0.5em] {\scriptsize{11:7}};
			\node[align=right, label, left=0.25em of ext.bin, yshift=0.5em] {\scriptsize{31:7}};
			\node[align=left, label, custyle, right=0.25em of cu.bregwrite, yshift=0.5em] {RegWrite};
			\node[align=left, label, custyle, right=0.25em of cu.bimmsrc, yshift=0.5em] {ImmSrc\textsubscript{1:0}};
			\node[align=left, label, custyle, right=0.25em of cu.balusrc, yshift=0.5em] {ALUSrc};
			\node[align=left, label, custyle, right=0.25em of cu.balucontrol, yshift=0.5em] {ALUControl\textsubscript{2:0}};
			\node[align=left, label, custyle, right=0.25em of cu.bmemwrite, yshift=0.5em] {MemWrite};
			\node[align=left, label, custyle, right=0.25em of cu.bresultsrc, yshift=0.5em] {ResultSrc};
			\node[align=left, label, custyle, right=0.25em of cu.bpcsrc, yshift=0.5em] {PCSrc};
			\node[align=left, label, right=0.25em of addpcplus4.bout, yshift=0.5em] {PCPlus4};
			\node[align=left, label, right=0.25em of addpctarget.bout, yshift=0.5em] {PCTarget};
			\node[align=left, label, right=0.25em of im.brd, yshift=0.5em] {Instr};
		\end{circuitikz}
	}
\end{center}


\subsection{Multi-Cycle RISC-V Processor}
\begin{center}
	\resizebox*{\textwidth}{!}{
		\begin{circuitikz}[font=\sffamily]
			\tikzset{
				label/.style={draw=none, inner sep=1pt, font=\small},
				comp/.style={align=center, no leads},
				custyle/.style={color=modblue},
			}

			%\node[anchor=center,inner sep=0, scale=1.1] at (0.6,2.25) {\includegraphics[width=1.5\textwidth]{/home/niklas/Pictures/Screenshots/mc.png}};


			% components
			\node[regfile, comp] (rf) {Register\\File};
			\node[datamem, comp, left=4cm of rf] (dm) {Instr/Data\\Memory};
			\node[alu,comp, right=3.75cm of rf.rd1, anchor=a] (alu) {ALU};

			\node[ctrlunitmc, comp, custyle, above=2.25cm of rf, xshift=-1.7cm] (cu) {Control\\Unit};

			\node[extend, comp, below=0.5cm of rf] (ext) {Extend};

			% TODO: regs with multiple inputs
			\node[reg, comp, right=2cm of rf.north, anchor=north, stacks=2] (regrfread) {};
			\node[3mux, comp, right=1.75cm of rf.brd1, anchor=bin2] (muxalusrca) {};
			\node[3mux, comp, right=2.5cm of rf.brd2, anchor=bin0] (muxalusrcb) {};
			\node[mux, comp, left=0.75cm of dm.ba, anchor=bout] (muxaddrnext) {};
			\node[reg, comp, left=0.75cm of muxaddrnext.bin0, enable input, anchor=bout] (regpcnext) {};
			\node[reg, comp, right=1.8cm of alu.bout, enable input, anchor=bin] (regaluresult) {};
			\node[3mux, comp, right=1.5cm of regaluresult.bout, anchor=bin0] (muxresult) {};
			\node[reg, comp, right=0.75cm of dm.brd, enable input, anchor=bin, yshift=1.5cm, stacks=2.5] (regreaddata) {};
			\node[reg, comp, right=0.75cm of dm.brd, anchor=bin, yshift=-3.75cm,] (regreaddata2) {};

			% special nodes
			% clks
			\node[align=center, label, above=0.25cm of regpcnext.clk] (clk1) {CLK};
			\draw[] (clk1) -- (regpcnext.clk);
			\node[align=center, label, above=0.25cm of regreaddata.clk] (clk2) {CLK};
			\draw[] (clk2) -- (regreaddata.clk);
			\node[align=center, label, above=0.25cm of regreaddata2.clk] (clk3) {CLK};
			\draw[] (clk3) -- (regreaddata2.clk);
			\node[align=center, label, above=0.25cm of regaluresult.clk] (clk4) {CLK};
			\draw[] (clk4) -- (regaluresult.clk);
			\node[align=center, label, above=0.25cm of regrfread.clk] (clk5) {CLK};
			\draw[] (clk5) -- (regrfread.clk);
			\node[align=center, label, above=0.25cm of rf.bclk] (clkrf) {CLK};
			\draw[] (clkrf) -- (rf.bclk);
			\node[align=center, label, above=0.25cm of dm.bclk] (clkdm) {CLK};
			\draw[] (clkdm) -- (dm.bclk);
			\node[align=center, label, custyle, above=0.25cm of cu.bclk] (clkcu) {CLK};
			\draw[custyle] (clkcu) -- (cu.bclk);

			% constants
			\node[align=right, label, left=0.25cm of muxalusrcb.bin2] (const4) {4};
			\draw[] (const4) -- (muxalusrcb.bin2);

			% connections
			% register file
			\draw[] (rf.brd1) -- (rf.brd1 -| regrfread.bin);
			\draw[] (rf.brd2) -- (rf.brd2 -| regrfread.bin);
			\draw[] (rf.brd2 -| regrfread.bout) -- ++(0.5, 0) -- ++(0, -1.75) -- ++(-10.25, 0) |- (dm.bwd);
			\draw[] (rf.brd1 -| regrfread.bout) -- (muxalusrca.bin2);
			\draw[] (rf.brd2 -| regrfread.bout) -- (muxalusrcb.bin0);

			% alu
			\draw[] (muxalusrca.bout) --++(1, 0) |- (alu.ba);
			\draw[] (muxalusrcb.bout) --++(0.25, 0) |- (alu.bb);
			\draw[] (alu.bout) -- (regaluresult.bin);
			\draw[] (alu.bzero) -- ++(1, 0) -- ++(0, 2.85)-- ++(-10.75,0)  |- (cu.bzero) ;

			% instruction/data memory
			\draw[] (dm.brd) --++(0.3,0) |- (regreaddata2.bin);
			\draw[] (dm.brd) --++(0.3,0) -- (dm.brd -| regreaddata.bin);
			\draw[] (regreaddata2.bout) --++(14,0) |- (muxresult.bin1);
			\draw[] (muxaddrnext.bout) -- (dm.ba);

			\draw[] (dm.brd -| regreaddata.bout) -- ++(1.1, 0) |- (cu.bop);
			\draw[] (dm.brd -| regreaddata.bout) -- ++(1.1, 0) |- (cu.bfunct3);
			\draw[] (dm.brd -| regreaddata.bout) -- ++(1.1, 0) |- (cu.bfunct7);
			\draw[] (dm.brd -| regreaddata.bout) -- ++(1.1, 0) |- (rf.ba1);
			\draw[] (dm.brd -| regreaddata.bout) -- ++(1.1, 0) |- (rf.ba2);
			\draw[] (dm.brd -| regreaddata.bout) -- ++(1.1, 0) |- (rf.ba3);
			\draw[] (dm.brd -| regreaddata.bout) -- ++(1.1, 0) |- (ext.bin);

			% extend unit
			\draw[] (ext.bout) --++(1.75, 0) |- (muxalusrcb.bin1);

			% control unit
			\draw[custyle] (cu.bregwrite) -| (rf.bwe3);
			\draw[custyle] (cu.bimmsrc) -- ++(2.65, 0) -- ++(0, -6.125) -| (ext.bctrl);
			\draw[custyle] (cu.balusrca) -| (muxalusrca.bsel);
			\draw[custyle] (cu.balusrcb) -| (muxalusrcb.bsel);
			\draw[custyle] (cu.balucontrol) -| (alu.bctrl);
			\draw[custyle] (cu.bresultsrc) -| (muxresult.bsel);
			\draw[custyle] (regreaddata.ben) -- ++(0, -0.25) -- ++(-0.4, 0) |- (cu.birwrite);
			\draw[custyle] (cu.bmemwrite) -| (dm.bwe);
			\draw[custyle] (cu.badrsrc) -| (muxaddrnext.bsel);
			\draw[custyle] (regpcnext.ben) -- ++(0, -0.25) -- ++(-0.4, 0) |- (cu.bpcwrite);

			% additional connections
			\draw[] (regpcnext.bout) -- (muxaddrnext.bin0);
			\draw[] (muxresult.bout) -- ++(0.25, 0) -- ++(0, -3.5) -- ++ (-21.5,0) |- (regpcnext.bin);
			\draw[] (muxresult.bout) -- ++(0.25, 0) -- ++(0, -3.5) -- ++ (-20,0) |- (muxaddrnext.bin1);
			\draw[] (muxresult.bout) -- ++(0.25, 0) -- ++(0, -3.5) -- ++ (-13.5,0) |- (rf.bwd3);
			% TODO: align with result signal
			\draw[] (regpcnext.bout) -- ++(0.55, 0) -- ++(0, 2.4) -- ++ (11.25,0) |- (muxalusrca.bin0);
			\draw[] (regpcnext.bout) -- ++(0.55, 0) |- (regreaddata.bin);
			\draw[] (regreaddata.bout) -- ++(6.75, 0) |- (muxalusrca.bin1);
			\draw[] (regaluresult.bout) -- (muxresult.bin0);
			\draw[] (alu.bout) -- ++(1,0) |- (muxresult.bin2);

			% labels
			\node[align=right, label, left=0.25em of alu.ba, yshift=0.5em] {SrcA};
			\node[align=right, label, left=0.25em of alu.bb, yshift=0.5em] {SrcB};
			\node[align=left, label, right=0.25em of alu.bout, yshift=0.5em] {ALUResult};
			\node[align=left, label, right=0.25em of alu.bzero, yshift=0.5em] {Zero};
			\node[align=left, label, anchor=south west, xshift=0.3cm, rotate=-90] at (rf.brd2 -| dm.brd) {ReadData};
			\node[align=left, label, anchor=south west, xshift=0.5cm, rotate=-90] at (rf.brd2 -| regrfread.bout) {WriteData};
			\node[align=left, label, anchor=west, yshift=0.5em] at (dm.brd -| regreaddata.bout) {Instr};
			\node[align=left, label, right=0.25em of ext.bout, yshift=0.5em] {ImmExt};

			\node[align=left, label, anchor=west, xshift=1.1cm, yshift=0.5em] at (regreaddata.bout |- cu.bop) {\scriptsize{6:0}};
			\node[align=left, label, anchor=west, xshift=1.1cm, yshift=0.5em] at (regreaddata.bout |- cu.bfunct3) {\scriptsize{14:12}};
			\node[align=left, label, anchor=west, xshift=1.1cm, yshift=0.5em] at (regreaddata.bout |- cu.bfunct7) {\scriptsize{30}};
			\node[align=left, label, anchor=west, xshift=1.1cm, yshift=0.5em] at (regreaddata.bout |- rf.ba1) {\scriptsize{19:15}};
			\node[align=left, label, anchor=west, xshift=1.1cm, yshift=0.5em] at (regreaddata.bout |- rf.ba2) {\scriptsize{24:20}};
			\node[align=left, label, anchor=west, xshift=1.1cm, yshift=0.5em] at (regreaddata.bout |- rf.ba3) {\scriptsize{11:7}};
			\node[align=left, label, anchor=west, xshift=1.1cm, yshift=0.5em] at (regreaddata.bout |- ext.bin) {\scriptsize{31:7}};

			\node[align=left, label, custyle, right=0.25em of cu.bregwrite, yshift=0.5em] {RegWrite};
			\node[align=left, label, custyle, right=0.25em of cu.bimmsrc, yshift=0.5em] {ImmSrc\textsubscript{1:0}};
			\node[align=left, label, custyle, right=0.25em of cu.balusrca, yshift=0.5em] {ALUSrcA\textsubscript{1:0}};
			\node[align=left, label, custyle, right=0.25em of cu.balusrcb, yshift=0.5em] {ALUSrcB\textsubscript{1:0}};
			\node[align=left, label, custyle, right=0.25em of cu.balucontrol, yshift=0.5em] {ALUControl\textsubscript{2:0}};
			\node[align=left, label, custyle, right=0.25em of cu.bresultsrc, yshift=0.5em] {ResultSrc\textsubscript{1:0}};
			\node[align=right, label, custyle, left=0.25em of cu.birwrite, yshift=0.5em] {IRWrite};
			\node[align=right, label, custyle, left=0.25em of cu.bmemwrite, yshift=0.5em] {MemWrite};
			\node[align=right, label, custyle, left=0.25em of cu.badrsrc, yshift=0.5em] {AdrSrc};
			\node[align=right, label, custyle, left=0.25em of cu.bpcwrite, yshift=0.5em] {PCWrite};
			\node[align=right, label, left=0.25cm of regpcnext.bin, yshift=0.5em] {PCNext};
			\node[align=left, label, right=0.25em of muxaddrnext.bout, yshift=0.5em] {Adr};
			\node[align=left, label, right=0em of regpcnext.bout, yshift=0.5em] {PC};
			\node[align=left, label, right=0em of regreaddata.bout, yshift=0.5em] {OldPC};
			\node[align=left, label, right=0em of regreaddata2.bout, yshift=0.5em] {Data};
			\node[align=left, label, right=0em of regaluresult.bout, yshift=0.5em] {ALUOut};
			\node[align=left, label, left=0.5cm of muxalusrca.bin2, yshift=0.5em] {A};
		\end{circuitikz}
	}
\end{center}

\end{document}

\end{document}

